\documentclass[10pt,t]{beamer}
\usepackage{msvcommon}

%Imported from Thesis Report

\usepackage{amssymb}
\usepackage{amsmath}
\usepackage{float}
\usepackage{subfig}
\usepackage[titletoc]{appendix}
\usepackage{siunitx}

\usepackage{listings}
\usepackage{soul,xcolor}\lstset{escapeinside={(*@}{@*)}}

% User defined Listing
\lstdefinestyle{DOS}
{
    backgroundcolor=\color{white},
    basicstyle=\scriptsize\color{black}\ttfamily
}

\usepackage{hyperref}
\usepackage{pdfpages}
\usepackage{rotating}
\usepackage{pdflscape}
\usepackage[acronym,nomain,nonumberlist,toc,shortcuts]{glossaries}


% Packages for Jonas Maas's Thesis Tikz block diagrams
\usepackage{tikz}
\usetikzlibrary{arrows.meta, calc}
\newcommand{\B}[1]{\boldsymbol{#1}}
\newcommand{\Bhat}[1]{\boldsymbol{\hat{#1}}}
\newcommand{\e}[1]{\emph{#1}}

\usepackage{caption}

% -------------------------------------------------------------------------------
% ** Input Encoding
% - utf8x is recommended at MSV
% - Some comments on encoding can be found here:
%   https://groups.google.com/forum/?fromgroups=#!msg/comp.text.tex/4LC-xODb-LU/1Bd5UZOMNM4J
%
\ifxetexorluatex
  % XeTeX and LuaTeX support utf8 by default
\else
  % File encoding utf8x
  \usepackage{ucs}
  \usepackage[utf8x]{inputenc}
\fi
%
%--------------------------------------------------------------------------------
% ** Language Settings and Hyphenation Patterns
%
% Note: final argument to babel sets the main language
%
\ifxetex%
\usepackage{polyglossia}
  \setmainlanguage{english}
  \setotherlanguage{german}
\else
  \usepackage[ngerman,english]{babel}
\fi


%--------------------------------------------------------------------------------
% ** MSV style defintions for beamer presentations
%
\usepackage[colorblock]{msvpresentation}
%-------------------------------------------------------------------------------
% ** Options for msvpresentation (all options are disabled by default)
% 1) boolean options for msvpresentation package:
%  | bw             | black and white color scheme                             |
%  | colorblock     | use non-white background colors for blocks               |
%  | frameblock     | use colored borders for blocks                           |
%
%--------------------------------------------------------------------------------
% ** Setup look of presentation
%
% MSV style should be compatible with the beamer themes
%
\useinnertheme{default}
\useoutertheme{default}

% Colors:
%\usecolortheme[named=TUMDarkerBlue]{structure} % variant: TUMBlack
\usecolortheme[named=TUMBlack]{structure} % variant: TUMBlack
\setbeamercolor{frametitle}{fg=TUMBlack} % variant: TUMDarkerBlue
\setbeamercolor{normal text}{fg=TUMBlack,bg=TUMWhite} % font/background
\setbeamercolor{alerted text}{fg=TUMBeamerRed,bg=TUMWhite} % alert boxes
\setbeamercolor{example text}{fg=TUMBeamerGreen,bg=TUMWhite} % example boxes

% Headers, Footers, Titlepage:
%   Note: can adjust height of the frame title with \raisetitle and \lowertitle
\setbeamertemplate{headline}[tum] % no, tum, msvtum
\setbeamertemplate{footline}[authortitle] % minimal, author, authortitle
\setbeamertemplate{title page}[tum][left] % with clock tower
%\setbeamertemplate{title page}[default][left,sep=0pt] % without clock tower

%--------------------------------------------------------------------------------
% ** Presentation title, author, etc.
%
\title{Experimental Evaluation of Machine Learning based Wireless Communication Algorithms}
\author[Karthik Sukumar (TUM)]{Karthik Sukumar}
\institute{\MSVname}
\date{07. Sep 2020}

\begin{document}
{ % use custom header/footer for title page
\setbeamertemplate{headline}[tum]
\setbeamertemplate{footline}[no]
\begin{frame}
\titlepage
\end{frame}
}

\section{Introduction}
\subsection{A Typical Slide}

%Intro Page

\begin{frame}{Introduction}

    \begin{itemize}
        \item Channel Estimation is vital for any Wireless systems
        \item Required to revert the channel propogation effects

            \pause

        \item For perfect Channel estimation, pilot symbols are to be placed on all sub carriers
        \item For massive MIMO $\Rightarrow$ No. of Antennas $\propto$ Pilot overhead

            \pause
        \item Current reseach for potential solutions include
            \begin{itemize}
                \item Model based channel estimation % where a 3D model of the environment is saved and ray tracing is used to find the Channel
                \item Machine learning based algorithms to find the channel
            \end{itemize}
    \end{itemize}

%\bigskip % Use \bigskips between paragraphs (do not set \parskip as this distorts block environments)

\end{frame}

\begin{frame}{Summary of Thesis}
    \begin{itemize}
        \item Setup a MIMO Test Jig
            \pause
        \item Collect real world experimental Tx-Rx data from the MIMO Setup
            \pause

        \item Use the data to train a Inverted Neural Network as shown below
            \bigskip

            \begin{center}
                \begin{figure}[H]
                    \resizebox{8cm}{!}{
                        \begin{tikzpicture}
                            \tikzstyle{network} = [rectangle, draw, minimum height = 15mm, minimum width = 20mm, line width = 0.5mm]

                            \node (a) at (0, 0) {$\Bhat{x}_j$};
                            \node[network] at (3, 0) (b) {Singlestream ML};
                            \node[network] at (7, 0) (c) {$\text{INN}^{-1}$};
                            \node (d) at (11, 0) {$\B{y}_j = \B{Hx}_j + \B{n}_j$};

                            \draw[->, line width = 0.5mm] (d) to (c);
                            \draw[->, line width = 0.5mm] (c) to node [above] {$\Bhat{x}_{\text{estim}, j}$} (b) ;
                            \draw[->, line width = 0.5mm] (b) to (a);

                        \end{tikzpicture}
                    }
                    \caption{ML Model for learning the network}
                    \label{fig:MLModel}
                \end{figure}
            \end{center}


            %\begin{itemize}
            %    \item Machine learning based channel prediction algorithms
            %\end{itemize}
    \end{itemize}
\end{frame}

\begin{frame}{Summary of Thesis}

    \begin{itemize}
        \item LTE Fundamentals
        \item Possible Options for a MIMO Setup
        \item Chosen Experimental Setup
        \item Results
        \item Conclusions and future work
    \end{itemize}
\end{frame}

\begin{frame}{LTE Fundamentals}
\end{frame}

\begin{frame}{Possible Options for a MIMO Setup}
\end{frame}

\begin{frame}{Experimental Setup}
\end{frame}

\begin{frame}{Demo}
\end{frame}

\begin{frame}{Results}
\end{frame}


{
    \begin{frame}{A Slide with a different header}
        Some text, possibly \textbf{bold} or \alert{highlighted}.

        \begin{itemize}
            \item Bullet points
            \item Second point
                \begin{itemize}
                    \item Sub-bullet
                \end{itemize}
        \end{itemize}

        \pause
        \begin{enumerate}
            \item First
            \item Second
            \item Third
        \end{enumerate}

    \end{frame}
}

{\lowertitle\lowertitle
\setbeamertemplate{headline}[msvtum]
\begin{frame}{Frame with highlight boxes and alternative header}

    \begin{block}{Important Result\footnotemark[1]}
        The following holds:\footnotemark[2]
        \begin{equation*}
            E = mc^2
        \end{equation*}
    \end{block}
    \footnotetext[1]{This is a footnote in a block title.}
    \footnotetext[2]{This is a footnote in a block body.}

    \pause

    \begin{center}
        $\Downarrow$
    \end{center}

    \begin{alertblock}{}
        Can also be used without a title.
        Three color types are available: block (blue), alertblock (red), and exampleblock (green).
        Spacing may need some manual adjustment if formulas are included at the top or bottom of blocks.
    \end{alertblock}

\end{frame}
}

{
\setbeamertemplate{headline}[no]
\begin{frame}{Some math}
    \begin{itemize}
        \item We consider some simple formulas, e.g. $\max(0,1) = 1$
        \item Complicated formula: $h_J(y) = \sum_{K\subset N}\int_{\mathbb{R}}g_K(x,y)f_J(x)dx$
        \item This looks slightly weired since math fonts are smaller than text fonts.
    \end{itemize}

    However, this does not really affect equations.
    \begin{equation*}
        h_J(y) = \sum_{K\subset N}\int_{\mathbb{R}}g_K(x,y)f_J(x)dx
    \end{equation*}
\end{frame}
}

\begin{frame}{}
    No title on this frame.\footnote{Smith et al., 2100: ``Title of a paper that will be written in the future'', \itshape IEEE Trans Inf. Theory}

    \begin{exampleblock}{Example Block}
        Spacing around blocks is minimal (if option frameblock is used).
        Extra spaces, e.g. \texttt{vskip} or \texttt{vspace}, should be used.
    \end{exampleblock}

    Text below block.

\end{frame}


\section{Main Part}


\begin{frame}{TUM Colors}
In diagrams and plots only use the following colors:
\begin{itemize}
\item Black, White
\item \textcolor{TUMBeamerYellow}{Yellow, RGB 255/180/000}
\item \textcolor{TUMBeamerOrange}{Orange, RGB 255/128/000}
\item \textcolor{TUMBeamerRed}{Red, RGB 229/052/024}
\item \textcolor{TUMBeamerDarkRed}{Dark Red, RGB 202/033/063}
\item \textcolor{TUMBeamerBlue}{Blue, RGB 000/153/255}
\item \textcolor{TUMBeamerLightBlue}{Light Blue, RGB 065/190/255}
\item \textcolor{TUMBeamerGreen}{Green, RGB 145/172/107}
\item \textcolor{TUMBeamerLightGreen}{Light Green, RGB 181/202/130}
\end{itemize}
\end{frame}



\section{Conclusion}


\begin{frame}{Final Slide}
Add some vertical space.

\vspace{8mm}
Increase spacing between bullet points:

\vspace{5mm}
\begin{itemize}\addtolength{\itemsep}{5mm}
\item More information can be found in the \alert{beamer user guide}
\item Use \texttt{pdflatex} to compile the source
\item Have fun creating your slides!
\end{itemize}
\end{frame}


\end{document}
