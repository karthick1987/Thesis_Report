%-------------------------------------------------------------------------------
% * MSV Report Class Template and Documentation
\documentclass[twoside]{msvreport}%
% - msvreport is an extension of the standard report class and passes
%   all options to report
% - default options are as in report class with exception of a4paper,
%   openright, and 11pt
% - msvcommon package is loaded and provides TUM color definitions,
%   logos, etc
%
%-------------------------------------------------------------------------------
% ** Options for msvreport (default options are marked (*))
% 1) boolean options for msvreport class:
%  | rmheads (*)    | use roman for headlines and contents                     |
%  | sfheads        | use sans-serif for headlines and contents                |
%  | chapterprefix  | print Chapter prefix in first chapter page               |
%  | bwchapters (*) | print chapter names in black                             |
%  | bluechapters   | print chapter names in blue                              |
%  | headrule       | print horizontal line in header                          |
%  | widetext (*)   | use a wide text body                                     |
%  | narrowtext     | use a narrow text body                                   |
%  | cmyk (*)       | use CMYK colors optimized for printing                   |
%  | print          | alias for cmyk                                           |
%  | rgb            | use RGB colors optimized for monitors                    |
%  | lores (*)      | use low-resolution TUM Banner                            |
%  | hires          | use high resolution TUM Banner                           |

%
% 2) Key/Value options for msvreport class
%  - coverstyle=(none|bw|banner|flags)
%    | none (*) | do not generate cover page                                               |
%    | color    | use color style for cover page                                           |
%    | bw       | use black/white style for cover page (eg for printing on colored covers) |
%    | banner   | use simple banner without flags for cover page (deprecated)              |
%    | flags    | use TUM flag banner for cover page (deprecated)                          |
%
%  - titlestyle=(flags|none|bw|banner)
%    same as coverstyle but for title page
%
%  - copyright=(none|cover|title)
%    | none (*) | do not print copyright notice           |
%    | cover    | print copyright notice after cover page |
%    | title    | print copyright notice after title page |
%
%
%-------------------------------------------------------------------------------
% ** MSV Font Package
% - Standard MSV Report fonts are Times for text and Computer Modern for maths
% - Use option timesmath to enable Times font also in math mode (only pdftex)
% - Do not use msvfonts package if you want to use your own fonts
% - Note that msvfonts loads amsmath package
%
\usepackage{msvfonts}
%
% *** Options for msvfonts
% 1) boolean options for msvfonts package:
% | timestext (*) | use times font for text                                        |
% | fancytext     | use garamond font for text                                     |
% |             +-+                                                                |
% | cmrmath (*) | use computer modern font for text and operators in math-mode     |
% | timesmath   | use times font for text and operators in math mode               |
% |             |   WARNING: timesmath in LuaLaTeX uses unicode-math, which does   |
% |             |   not support the bm package. Use \symbfit and \symbf instead.   |
%
%
% -------------------------------------------------------------------------------
% ** Input Encoding
% - utf8x is recommended at MSV
% - Some comments on encoding can be found here:
%   https://groups.google.com/forum/?fromgroups=#!msg/comp.text.tex/4LC-xODb-LU/1Bd5UZOMNM4J
%
\ifxetexorluatex
  % XeTeX and LuaTeX support utf8 by default
\else
  % File encoding utf8x
  \usepackage{ucs}
  \usepackage[utf8x]{inputenc}
\fi
%
%--------------------------------------------------------------------------------
% ** Language Settings and Hyphenation Patterns
%
% Note: final argument to babel sets the main language
%
\ifxetex%
  \usepackage{polyglossia}
  \setmainlanguage{english}
  \setotherlanguage{german}
%  \setmainlanguage{german}
%  \setotherlanguage{english}
\else
  \usepackage[ngerman,english]{babel} 
\fi
%
%--------------------------------------------------------------------------------
% ** Custom Preamble
% - Start your custom preamble here
% - Use the space before \begin{document} to
%   - Load packages \usepackage{...}
%   - Define custom math operators (DeclareMathOperator)
% - Everything else (including newcommands) starts
%   below \begin{document}

% Here is a list of commonly used packages
% | amsmath   | Many useful tools for typesetting mathematics        |
% | amssymb   | Loads symbol fonts for mathematics, eg \mathbb       |
% | amsthm    | \newtheorem,\theoremstyle commands etc               |
% | thmtools  | Some advanced options for theorem customization      |
% | mathtools | Many useful tools for typesetting maths      |
% | bm        | Bold math (Warning: use only if bold fonts available |
% | algorithm | Tools to typeset algorithms                          |
% | booktabs  | Nice looking tables                                  |
% | enumerate | Enumerated lists                                     |
% | subfig    | Subfigures with common caption                       |
% | caption   | Control caption font                                 |
% | relsize   | Some useful relative font size commands              |
% | xspace    | Control white space after abbreviations              |
% | csquotes  | Quotations according to language selection           |

%-------------------------------------
% random.sty: for sans-serif math fonts
% use with:
%   \usepackage[greek,ngerman,english]{babel}
% does not work with 'timesmath' option in msvfonts
%


\usepackage{blindtext}

\begin{document}
%--------------------------------------------------------------------------------
% ** Custom Commands
% Put your \newcommand's here or \input them from a file
%
%
%--------------------------------------------------------------------------------
% ** Title and Author Information
%
\title{Experimental Evaluation of Machine Learning based Wireless Communication Algorithms}
\author{Karthik Sukumar}

\msvdoctype{Master Thesis}

\msvcovertext{
Supervisor: Prof. Wolfgang Utschick\\[1.5em]
Submission: Xxx xx, 2020
}

%--------------------------------------------------------------------------------
% ** Title Page
%
\frontmatter
\maketitle
%
%--------------------------------------------------------------------------------
% ** Optional: Abstract
%
\begin{abstract}
Put your abstract text here.
\end{abstract}
%
%--------------------------------------------------------------------------------
% ** Table of Contents
%
\tableofcontents
%
%
%--------------------------------------------------------------------------------
% ** First Chapter
%
\mainmatter
\chapter{Example with Citations}
\blindtext

\section{Citation and Equation}
If we write a floating point definition without suffixes$\mathrm{suffixes}\operatorname{suffixes}$ as in $y = \sin(x)$ sin(x), we obtain inline math symbols~\cite{Aldroubi01,Bazaraa06,WeStSh06}.
\begin{equation}
\prod_{j=1}^J\sum_{i=1}^\infty \int_{-\infty}^\zeta e^{\xi^2/\nu}d\xi \int_{S^1} \mathcal{R}f_i^j(\omega) d\omega
\end{equation}
\begin{equation}
\left(\frac{\sin(x)}{\pi x} <\leq=\geq>|\|\sim+-\pm \| z^Hz\|\right)\quad \forall \zeta \in R
\end{equation}%

\blindtext[8]

\section{This is a Section}
\blindtext[1]

\subsection{This is a Subsection}
\blindtext[1]

\subsubsection{This is a Subsubsection}
\blindtext[1]

\paragraph{This is a Paragraph}
\blindtext[1]

\subparagraph{This is a Subparagraph}
\blindtext[1]


\chapter{Check Text Height}
\blindtext[40]

\blindmathtrue
\Blinddocument
\blinddocument


%--------------------------------------------------------------------------------
% ** Appendix
%
\appendix
\blinddocument

\cleardoublepage
% \bibliographystyle{unsrt}
%--------------------------------------------------------------------------------
% ** References 
\bibliographystyle{IEEEtran}
\refstepcounter{chapter}
\addcontentsline{toc}{chapter}{\bibname}
\bibliography{IEEEabrv,examplebibfile}

% \cleardoublepage
% \refstepcounter{chapter}
% \addcontentsline{toc}{chapter}{\indexname}
% \printindex
% \end{comment}
\end{document}
