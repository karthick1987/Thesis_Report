\chapter{Potential Hardware Setups}
\label{ch:PotenHWSetup}

Measurements of MIMO channel can be achieved in multiple methods. This chapter discusses some of the potential approaches which were implemented and elaborates each of their advantages and disadvantages.

\section{Software Defined Radios USRP}\label{sec:USRP}

USRP is a Software Defined Radio (SDR) designed by National Instruments that enables quick prototyping of different wireless applications. It is aimed at anyone from hobbists, research labs, universities, etc... or anyone interested in evaluating custom algorithms. The SDR used here is a USRP2940 specifications of which are described in Table \ref{tb:USRP}

\begin{table}[H]
    \begin{center}
        \begin{tabular}{|l|c|}
        \hline
            Model                   & USRP2940          \\ \hline
            Baseband Bandwidth      & 40MHz             \\ \hline
            RF-Operating Frequency  & 50MHz-2200MHz     \\ \hline
            FPGA                    & Kintex-7 410T     \\ \hline
            No of Transmitters      & 2                 \\ \hline
            No of Receivers         & 2                 \\ \hline
            Connectivity            & MXIe, Ethernet    \\ \hline
            Oscillator              & Internal Crystal  \\ \hline
            ADC/DAC                 & 14 (For Rx)/16 (For Tx) bit         \\ \hline
            Frequency Accuracy      & 2.5 ppm           \\ \hline
            Maximum Power Output    & 20dBm             \\ \hline
        \end{tabular}
    \end{center}
    \caption{USRP SDR Product details}
    \label{tb:USRP}
\end{table}

%\begin{table}[H]
%    \begin{center}
%        \begin{tabular}{|c|c|}
%            \hline
%            Parameter & Value \\ \hline
%            RBW & 20kHz \\ \hline
%            VBW & 50kHz \\ \hline
%            SWP Time & 50ms  \\ 
%            \hline
%        \end{tabular}
%    \end{center}
%    \caption{Spectrum analyser settings for the transmit power tests}
%    \label{}
%\end{table}

\section{MIMO Application Framework}\label{sec:MIMOAFW}
\section{LTE Application Framework}\label{sec:LTEAFW}



%\begin{figure}[!htb]
%    \centering
%    \includegraphics[width=7cm]{../ReportImages/TxPwrSetup.png}
%    \caption{Setup for measuring the transmit power}%
%    \label{fig:TxPwrSetup}%
%\end{figure}

