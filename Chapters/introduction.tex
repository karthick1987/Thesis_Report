\chapter{Introduction}
\label{ch:intro}

Channel estimation is an important part not only in LTE but for any Orthogonal Frequency Division Multiplexing (OFDM) system in general. It is neccesary to invert the channel propogation effects to reduce bit error rate and improve data throughput \cite{ChEst,CHEstPilotBased}. Channel estimation in OFDM systems is based on estimating the Channel Frequency Responce (CFR). For a perfect estimation every frequency and time resource block must have a reference symbol (pilot), but this leads to a high pilot overhead. To save on the bandwidth used for pilot transmission LTE inserts them sparsly in the 2D OFDM grid both in the time and frequency axis. The CFR for the complete grid is obtained by a 2D interpolation. The spacing in the time domain of the pilot symbols is called the coherence time, which is the minimum time for which the channel is expected to remain constant. The complexity increases when there are multiple antennas and in this case the channel needs to be estimated for each antenna i.e, pilots are needed for every antenna. This a challenge for massive MIMO applications where several antennas are used and each antenna needs to have pilots inserted in the 2D time frequency grid. There is a limited bandwidth for pilots to be filled in the grid and with massive MIMO the pilot overhead scales linearly with the number of antennas used.

Furthermore the complexity of Downlink is higher than in the Uplink case as the UEs are not expected to carry a massive MIMO transmitter or receiver. Current alternatives to pilot based channel estimation are model based channel estimation, where the environment is built in software and ray tracing employed to determine the channel. This is very resource intensive and needs to be updated in a dynamic environment. Research is also looking into machine learning applications for channel estimation. 

In parallel to this thesis an inverse neural network (INN) based machine learning algorithm has been developed \cite{JMMLINN}, and simultaed data was used to train the network. The output results of the trained network was impresive and the next step was to use real world MIMO data to evaluate the performance of the neural network.

This thesis aims to setup a functioning experimental MIMO jig in order to use the empirical data to evaluate the above mentioned machine learning algorithm that inverts the channel propogation effects. This is done by training an inverse neural network (INN) using experimental data from a MIMO test jig. \\

This work has the following structure: \\

Chapter \ref{ch:sysmodel} contains the system model description and the basics MIMO transceivers.\\
Chapter \ref{ch:ChEst} contains the fundamentals of channel estimation and the different detaied algorithms descriptions. \\
Chapter \ref{ch:PotenHWSetup} contains the different possible solutions for setting up a MIMO testjig with the advantages and disadvantages of every approach. \\
Chapter \ref{ch:ExSetup} contains the detailed description of the final chosen approach for the MIMO setup. \\
Chapter \ref{ch:results} discusses the results of the experiments that were run with the chosen approach. \\
Chapter \ref{ch:conOutlook} concludes with potential improvements and tweaks to the existing setup.
